% !TeX program = xelatex
% !TeX encoding = UTF-8
\documentclass[a4paper]{article}

\usepackage[UTF8]{ctex}
\usepackage[a4paper,margin=1in]{geometry}
\usepackage{graphicx}
\usepackage{listings}
% \usepackage{listings-rust}
\usepackage{longtable}
\usepackage{booktabs}
\usepackage{hyperref}
\usepackage{fancyhdr}
\usepackage{lastpage}
\usepackage{color}
\usepackage{indentfirst}        % 首段缩进
\setlength{\parindent}{2em}     % 缩进2字符
\usepackage{zhnumber}           % 中文编号
\usepackage[dvipsnames,table,xcdraw]{xcolor}
\usepackage{adjustbox}
\usepackage{subcaption}
\usepackage{float}
\usepackage{amsmath}
\usepackage{amssymb}
\usepackage[many]{tcolorbox}
% \newfontfamily{\enfont}{Arial}  % 定义英文字体
\usepackage{enumitem}
\usepackage{xeCJK}
\usepackage[T1]{fontenc}
\usepackage{lmodern}
\usepackage{inconsolata} % 使用更美观的等宽字体 Inconsolata
\usepackage{multirow} % 用于表格中的多行合并
\usepackage{tikz} % 添加 TikZ 宏包用于绘制流程图
\usetikzlibrary{positioning} % 添加定位库以支持 'of' 语法
\usepackage{algorithm}
\usepackage{algorithmic}

% 全局设置(影响所有级别)
\setlist[itemize]{
    itemsep=2pt,
    topsep=4pt,
    parsep=1pt,
    leftmargin=3em
}

% 仅设置第一级(最高级)
\setlist[itemize,1]{
    label=\textcolor{orange}{$\blacktriangleright$},
    font=\color{blue!80!black}\small\bfseries,
    before={\color{blue!80!black}\small\bfseries}
}

% 重置第二级及更深级别(恢复默认样式)
\setlist[itemize,2]{
    label={\normalfont\textbullet},
    font=\normalfont\normalsize\color{black},
    before={\color{black}},
    leftmargin=*
}
\setlist[itemize,3]{
    label={\normalfont$\circ$},
    font=\normalfont\normalsize,
    before={},
    leftmargin=*
}


% 定义可分页的引用环境
\newtcolorbox[auto counter]{myquote}[1][]{
    % breakable,                     % 允许跨页分断
    % enhanced jigsaw,               % 优化分页处的边框
    colback=gray!5,                % 背景色
    colframe=gray!30,              % 边框色
    left=5mm, right=5mm,           % 左右内边距
    top=3mm, bottom=3mm,           % 上下内边距
    width=0.85\textwidth,         % 宽度
    center,                     % 居中
    arc=0mm,                       % 直角边框
    boxrule=0.5pt,                 % 边框线宽
    fontupper=\small\linespread{1.1}\selectfont, % 行距微调(1.1倍)
    % fontupper=\sffamily\linespread{1.05}\selectfont, % 无衬线 + 紧凑行距
    before upper={\parindent2em\color{black!40!blue}\itshape\small\kaishu}, % 深蓝色文本
    #1 % 允许额外参数
}

%\newcommand{\college}{中山大学计算机学院}
\newcommand{\projname}{计算机图形学 | HW3}
\newcommand{\reporttitle}{GLSL入门与Phong光照模型实现}
\newcommand{\stuno}{23336188}
\newcommand{\authorname}{欧阳易芃}
\newcommand{\major}{计算机科学与技术}
\newcommand{\adviser}{陶钧}
\newcommand{\startdate}{2025年1月10日}
\newcommand{\labenddate}{2025年1月16日}
\newcommand{\labroom}{超算307}

\pagestyle{fancy} % 使用 fancyhdr 风格
\fancyhf{}      % 清空默认的页眉页脚

% 设置页眉
\fancyhead[L]{\kaishu \projname}      % 左侧页眉显示项目名称
\fancyhead[C]{\kaishu \reporttitle}    % 中间页眉显示报告标题
\fancyhead[R]{\kaishu \authorname} % 右侧页眉显示作者姓名

% 设置页脚
\fancyfoot[C]{第 \thepage 页,共 \pageref{LastPage} 页} % 中间页脚显示页码

% 去除页眉页脚与正文之间的分隔线
\renewcommand{\headrulewidth}{0.4pt}
\renewcommand{\footrulewidth}{0pt}

% 设置文档背景颜色
\definecolor{mybgcolor}{RGB}{0, 0, 0} % 定义背景颜色
\definecolor{NavyBlue}{RGB}{0,0,128} % 添加颜色定义



% 全局设置lstlisting的样式
\lstset{
    language=Python, % 设置默认语言
    basicstyle=\footnotesize\ttfamily, % 设置字体族
    breaklines=true, % 自动换行
    keywordstyle=\bfseries\color{blue}, % 设置关键字为粗体,颜色为蓝色
    % morekeywords={}, % 设置更多的关键字,用逗号分隔
    % emph={}, % 指定强调词,如果有多个,用逗号隔开
    % emphstyle=\bfseries\color{Rhodamine},
    commentstyle=\itshape\color{green!50!black},
    stringstyle=\bfseries\color{PineGreen!90!black},
    columns=fixed,% 设置列宽
    numbers=left,
    numbersep=2em,
    numberstyle=\footnotesize\color{gray},
    frame=single,
    framesep=1em,
    showstringspaces=false,% % 不显示字符串中的空格
    keepspaces=true,
    backgroundcolor=\color{gray!5},
    % linewidth=0.8\linewidth, % 设置宽度为当前行宽的 80%
    xleftmargin=0.2\linewidth, % 左右边距自动调整
    xrightmargin=0.08\linewidth,
    % gobble=2, % 可选:移除代码前的缩进
    % postbreak=\mbox{\textcolor{red}{$\hookrightarrow$}\space}, % 可选:换行符样式
    captionpos=b, % 可选:标题位置
    aboveskip=1em, % 可选:上方间距
    belowskip=1em, % 可选:下方间距
}

% 定义Python语言的样式
\lstdefinelanguage{Python}{
    basicstyle=\footnotesize\ttfamily,
    breaklines=true,
    keywordstyle=\bfseries\color{blue},
    morekeywords={import,from,as,def,class,return,if,elif,else,for,while,in,try,except,with,lambda,print},
    emph={self,True,False,None},
    emphstyle=\bfseries\color{Rhodamine},
    commentstyle=\itshape\color{green!50!black},
    stringstyle=\bfseries\color{PineGreen!90!black},
}

% 定义交换机/路由器配置命令的样式
\lstdefinelanguage{HuaweiConfig}{
    basicstyle=\footnotesize\ttfamily,
    breaklines=true,
    keywordstyle=\bfseries\color{NavyBlue},
    morekeywords={system-view,sysname,interface,port,link-type,access,trunk,allow-pass,vlan,stp,mode,enable,ip,address,ospf,area,network,quit,display,brief,clock,datetime,traffic-filter,inbound,acl,rule,permit,deny,time-range,description,undo,shutdown,batch,edged-port,GigabitEthernet},
    emph={GE0/0/0,GE0/0/1,GE0/0/2,GE0/0/3,GigabitEthernet0/0/0,GigabitEthernet0/0/1,GigabitEthernet0/0/2,GigabitEthernet0/0/3,GigabitEthernet0/0/22,GigabitEthernet0/0/23,GigabitEthernet0/0/24,S1,S2,R1,R2,Huawei},
    emphstyle=\bfseries\color{Rhodamine},
    commentstyle=\itshape\color{green!50!black},
    stringstyle=\bfseries\color{PineGreen!90!black},
    comment=[l]{\#},
    moredelim=[s][\color{magenta}]{<}{>},
    moredelim=[s][\color{magenta}]{[}{]},
}

% 定义cmd命令行的样式
\lstdefinelanguage{CmdLine}{
    basicstyle=\footnotesize\ttfamily,
    breaklines=true,
    keywordstyle=\bfseries\color{NavyBlue},
    morekeywords={ipconfig,ping,tracert,netstat,arp,route,nslookup,ftp,telnet},
    emph={/all,/s,/d,/n,/r,/h},
    emphstyle=\bfseries\color{Rhodamine},
    commentstyle=\itshape\color{green!50!black},
    stringstyle=\bfseries\color{PineGreen!90!black},
}

% 定义GLSL语言的样式
\lstdefinelanguage{GLSL}{
    basicstyle=\footnotesize\ttfamily,
    sensitive=true,
    morekeywords={
        attribute, const, uniform, varying,
        layout, location,
        in, out, inout,
        float, int, void, bool, true, false,
        lowp, mediump, highp, precision,
        discard, return,
        mat2, mat3, mat4,
        vec2, vec3, vec4, ivec2, ivec3, ivec4, bvec2, bvec3, bvec4,
        sampler1D, sampler2D, sampler3D, samplerCube,
        sampler1DShadow, sampler2DShadow,
        struct
    },
    morekeywords=[2]{
        radians,degrees,sin,cos,tan,asin,acos,atan,
        pow,exp,log,exp2,log2,sqrt,inversesqrt,
        abs,sign,floor,ceil,fract,mod,min,max,clamp,mix,step,smoothstep,
        length,distance,dot,cross,normalize,faceforward,reflect,refract,
        matrixCompMult,
        lessThan,lessThanEqual,greaterThan,greaterThanEqual,equal,notEqual,
        any,all,not,
        texture1D,texture1DProj,texture1DLod,texture1DProjLod,
        texture2D,texture2DProj,texture2DLod,texture2DProjLod,
        texture3D,texture3DProj,texture3DLod,texture3DProjLod,
        textureCube,textureCubeLod,
        shadow1D,shadow2D,shadow1DProj,shadow2DProj,
        shadow1DLod,shadow2DLod,shadow1DProjLod,shadow2DProjLod,
        dFdx,dFdy,fwidth,
        noise1,noise2,noise3,noise4
    },
    morecomment=[l]{//},
    morecomment=[s]{/*}{*/},
    morecomment=[l][\color{magenta}]{\#}
}
% 自定义盒子
\newtcolorbox{warningbox}[1][]{
    colback=red!5!white,
    colframe=red!75!black,
    title=#1,
    fonttitle=\bfseries
}

\newtcolorbox{infobox}[1][]{
    colback=blue!5!white,
    colframe=blue!75!black,
    title=#1,
    fonttitle=\bfseries
}

\begin{document}


% 封面
\begin{titlepage}
    \centering
    
    \includegraphics[width=10cm]{img/SYSULogo.png}

    \vspace{1em}
    %{\Large \college \par}
    \vspace{1em}
    {\Large \kaishu \projname \par}
    \vspace{3em}

    {\fontsize{40pt}{42pt}\kaishu \selectfont \boldmath \reporttitle\par}
    \vspace*{\fill}

    \begin{center}
    {\Large
    \makebox[5em][s]{学号}:\underline{\makebox[15em][c]{\kaishu \stuno}}\\[1em]
    \makebox[5em][s]{姓名}:\underline{\makebox[15em][c]{\kaishu \authorname}}\\[1em]
    \makebox[5em][s]{专业}:\underline{\makebox[15em][c]{\kaishu \major}}\\[1em]
    \makebox[5em][s]{指导教师}:\underline{\makebox[15em][c]{\kaishu \adviser}}\\[1em]
    \makebox[5em][s]{起始日期}:\underline{\makebox[15em][c]{\kaishu \startdate}}\\[1em]
    \makebox[5em][s]{结束日期}:\underline{\makebox[15em][c]{\kaishu \labenddate}}\\[1em]
    \makebox[5em][s]{实验地点}:\underline{\makebox[15em][c]{\kaishu \labroom}}
    }
    \end{center}

    \vspace*{\fill}
\end{titlepage}

% 目录
\tableofcontents
\newpage

\section{绪论}

\subsection{研究背景与意义}
飞行模拟作为计算机图形学(Computer Graphics, CG)最古老也最硬核的应用领域之一,始终推动着实时渲染技术的发展。从早期的线框模型到如今照片级真实的《微软飞行模拟》,人们对于“飞行的渴望”从未停止。

在传统的游戏引擎开发流程中,构建一个庞大的开放世界通常需要耗费巨大的人力成本。美术师需要手动雕刻山脉、绘制河流、摆放植被。这种“手工打造”的方式虽然精细,但受限于存储空间和人力,地图的边界总是有限的。当玩家飞到地图边缘时,通常会遇到“空气墙”或者被强制传送回去,这极大地破坏了沉浸感。

近年来,\textbf{过程化内容生成(PCG)} 技术成为解决这一问题的银弹。通过数学算法(如噪声函数、分形几何、L-System等)自动生成内容,计算机可以在运行时动态创造出无穷无尽的世界。这种技术不仅极大地节省了存储空间(只需要存储几KB的代码和种子),还理论上消除了世界的边界。

本项目 Skyscape 正是基于这一理念的实践探索。通过单纯使用代码和数学公式,而非预制的 3D 资产,去构建一个无限延伸的自然世界。这不仅是对计算机图形学理论知识的综合运用,也是对软件工程架构设计能力的极大考验。

\subsection{国内外现状}
\subsubsection{地形渲染技术}
目前主流的地形渲染技术主要分为三类:
\begin{enumerate}
    \item \textbf{高度图地形(Heightmap Terrain)}:使用灰度图存储高度信息。优点是直观、易于美术编辑;缺点是难以表现悬崖、山洞等复杂几何结构,且受限于图片分辨率和纹理尺寸。
    \item \textbf{体素地形(Voxel Terrain)}:如《Minecraft》,将世界划分为立方体单元。优点是可以完全破坏、挖洞;缺点是内存占用巨大,除了特定美术风格外,难以表现真实山体。
    \item \textbf{过程化网格(Procedural Mesh)}:如《No Man's Sky》,基于噪声函数实时生成顶点数据。优点是无限、连续;缺点是计算量大,对算法优化要求高。
\end{enumerate}
本项目选择第三种方案,即基于过程化网格的技术路线。

\subsection{本文组织结构}
本文共分为八个章节,具体安排如下:
\begin{itemize}
    \item \textbf{第一章 绪论}:介绍项目背景、意义及技术选型。
    \item \textbf{第二章 数学与理论基础}:详细推导项目涉及的线性代数、变换矩阵、噪声算法及光照模型。
    \item \textbf{第三章 系统架构设计}:阐述 ECS 架构思想、类图设计及模块划分。
    \item \textbf{第四章 无限地形系统实现}:深入剖析 Chunk 算法、LOD 策略及法线计算。
    \item \textbf{第五章 渲染与着色技术}:讲解 Shader 编程、纹理混合及雾效实现。
    \item \textbf{第六章 物理与交互系统}:描述飞行控制逻辑及摄像机跟随算法。
    \item \textbf{第七章 测试与优化}:展示实验结果,分析性能瓶颈及解决方案。
    \item \textbf{第八章 总结与展望}:回顾项目成果,展望未来改进方向。
\end{itemize}

\newpage

\section{数学与理论基础}

计算机图形学的基石是数学。在 Skyscape 的开发过程中,涉及了大量的线性代数和微积分知识。本章将详细介绍这些数学工具及其在项目中的具体应用。

\subsection{齐次坐标与变换矩阵}
在 3D 图形学中,我们通常使用 4 维齐次坐标 $(x, y, z, w)$ 来表示 3D 空间中的点和向量。引入 $w$ 分量的主要原因是为了统一处理 \textbf{线性变换}(旋转、缩放)和 \textbf{仿射变换}(平移)。

对于一个 3D 点 $P(x, y, z)$,其齐次坐标为 $P'(x, y, z, 1)$。
对于一个 3D 向量 $V(x, y, z)$,其齐次坐标为 $V'(x, y, z, 0)$。

所有的变换都可以通过 $4 \times 4$ 的矩阵乘法来完成。我们在 GLSL 着色器和 C++ 端(使用 GLM 库)都大量使用了这些矩阵。

\subsubsection{模型矩阵 (Model Matrix)}
模型矩阵用于将物体从局部空间转换到世界空间。它通常由平移矩阵 $T$、旋转矩阵 $R$ 和缩放矩阵 $S$ 组合而成:
\begin{equation}
    M_{model} = T \cdot R \cdot S
\end{equation}
其中平移矩阵的形式为:
\begin{equation}
    T = \begin{bmatrix}
    1 & 0 & 0 & t_x \\
    0 & 1 & 0 & t_y \\
    0 & 0 & 1 & t_z \\
    0 & 0 & 0 & 1
    \end{bmatrix}
\end{equation}

在 Skyscape 中,飞机的模型变换就需要实时更新。每当玩家按下 AWSD 键控制飞机时,我们会更新飞机的旋转矩阵 $R$ 和位置向量(对应 $T$),从而在下一帧渲染出运动的效果。

\subsubsection{视图矩阵 (View Matrix)}
视图矩阵用于将世界空间的坐标转换到摄像机空间(观察空间)。这本质上是一个以摄像机位置为原点、摄像机朝向为坐标轴的逆变换。
OpenGL 中常用的 `LookAt` 矩阵推导如下:
设摄像机位置为 $P$,目标位置为 $T$,世界上方为 $U_{world}$。
1. 计算前向向量(Z轴,注意摄像机看向 -Z):$F = \text{normalize}(T - P)$。但通常 LookAt 定义 $D = P - T$ 为正 Z 轴。这里我们采用 OpenGL 标准,前向向量 $F = \text{normalize}(T - P)$,则 $Z_{cam} = -F$。
2. 计算右向量(X轴):$R = \text{normalize}(F \times U_{world})$。
3. 计算上向量(Y轴):$U = R \times F$。

则 LookAt 矩阵为旋转矩阵与平移矩阵的组合:
\begin{equation}
    M_{view} = \begin{bmatrix}
    R_x & R_y & R_z & 0 \\
    U_x & U_y & U_z & 0 \\
    -F_x & -F_y & -F_z & 0 \\
    0 & 0 & 0 & 1
    \end{bmatrix} \cdot 
    \begin{bmatrix}
    1 & 0 & 0 & -P_x \\
    0 & 1 & 0 & -P_y \\
    0 & 0 & 1 & -P_z \\
    0 & 0 & 0 & 1
    \end{bmatrix}
\end{equation}

在每一帧的渲染循环中,我们调用 `camera.GetViewMatrix()` 来获取这个矩阵,并将其传递给着色器的 `uniform mat4 view` 变量。

\subsubsection{投影矩阵 (Projection Matrix)}
投影矩阵负责将视锥体(Frustum)内的坐标映射到标准化设备坐标(NDC,范围 $[-1, 1]$)内的立方体中。透视投影矩阵引入了 $1/z$ 的缩放因子,从而产生了近大远小的透视效果。

投影矩阵的数学形式较为复杂,其通过定义 **视野角 (Field of View, FOV)**,**近平面 (Near Plane)** 和 **远平面 (Far Plane)** 来构建。

\begin{equation}
    M_{proj} = \begin{bmatrix}
    \frac{1}{aspect \cdot \tan(fov/2)} & 0 & 0 & 0 \\
    0 & \frac{1}{\tan(fov/2)} & 0 & 0 \\
    0 & 0 & -\frac{f+n}{f-n} & -\frac{2fn}{f-n} \\
    0 & 0 & -1 & 0
    \end{bmatrix}
\end{equation}

\subsection{光照模型理论}
为了模拟真实世界的光影,我们采用了经典的 \textbf{Phong 反射模型}。该模型是经验模型,虽不完全符合物理定律,但计算高效且效果尚可。

光照强度 $I$ 由三部分组成:
\begin{equation}
    I = I_{ambient} + I_{diffuse} + I_{specular}
\end{equation}

\subsubsection{环境光 (Ambient)}
环境光模拟的是光线在场景中经过无数次漫反射后的整体亮度。它是一个常数:
\begin{equation}
    I_{ambient} = k_a \cdot L_a
\end{equation}
其中 $k_a$ 是材质对环境光的反射系数,$L_a$ 是环境光强度。在本项目中,我们设置了一个微弱的蓝灰色环境光,以模拟天空的散射光。

\subsubsection{漫反射 (Diffuse)}
漫反射模拟的是粗糙表面对光线的均匀散射。根据 \textbf{兰伯特余弦定律 (Lambert's Cosine Law)},反射光强与入射光线方向 $L$ 和表面法线 $N$ 的夹角余弦成正比:
\begin{equation}
    I_{diffuse} = k_d \cdot \max(N \cdot L, 0) \cdot L_d
\end{equation}
注意这里的 $N$ 和 $L$ 必须是归一化向量。漫反射是地形的主要色彩来源,当太阳(光源)移动时,山体的向阳面变亮,背阳面变暗,从而产生了立体感。

\subsubsection{镜面光 (Specular)}
镜面光模拟的是光滑表面产生的强光斑。Phong 模型基于观察方向 $V$ 和反射方向 $R$ 的夹角计算:
\begin{equation}
    I_{specular} = k_s \cdot \max(V \cdot R, 0)^{\alpha} \cdot L_s
\end{equation}
其中反射向量 $R$ 可通过 $R = \text{reflect}(-L, N) = 2(N \cdot L)N - L$ 计算得到。
$\alpha$ 是 \textbf{反光度 (Shininess)},值越大,光斑越小越亮(模拟金属);值越小,光斑越散(模拟塑料)。
在 Skyscape 中,我们对水面使用了高光处理($\alpha=64$),使其在阳光下闪闪发光;而对草地和岩石则几乎不计算镜面光($\alpha=2$),以表现其粗糙质感。

\subsection{噪声算法原理}
柏林噪声(Perlin Noise)是 Ken Perlin 于 1983 年为电影《电子世界争霸战》开发的,他也因此获得了奥斯卡技术成就奖。它是生成自然纹理(如云层、火焰、山脉)的基础。

\subsubsection{一维噪声与插值}
普通的随机函数 `rand(x)` 生成的是离散的、跳跃的数值,形成的图像是“白噪声”,完全没有连续性。
柏林噪声的核心在于 \textbf{梯度 (Gradient)} 和 \textbf{插值 (Interpolation)}。

对于 1D 输入 $x$,我们找到其前后的整数点 $x_0 = \lfloor x \rfloor$ 和 $x_1 = x_0 + 1$。
在这些整数点上定义伪随机的梯度值(Gradient)。然后计算输入点 $x$ 距离这些整数点的距离向量,并与梯度做点积。
最后,为了保证二阶导数连续,通常使用 \textbf{缓动函数 (Ease Curve)}:
\begin{equation}
    f(t) = 6t^5 - 15t^4 + 10t^3
\end{equation}
代替简单的线性插值 $f(t) = t$。这样生成的曲线圆滑自然,没有尖锐的拐点。

\subsubsection{分形布朗运动 (FBM)}
单一频率的噪声看起来像是一个平滑的馒头山,缺乏细节。自然界的地形具有 \textbf{自相似性 (Self-similarity)},即在不同的尺度上看起来都很粗糙。
我们通过叠加多层噪声(Octaves)来模拟这种特性,这种技术称为分形布朗运动(FBM):
\begin{equation}
    \text{Height}(x, z) = \sum_{i=0}^{N-1} A \cdot b^i \cdot \text{noise}(f \cdot 2^i \cdot x, f \cdot 2^i \cdot z)
\end{equation}
其中:
\begin{itemize}
    \item \textbf{Octaves ($N$)}:叠加的层数。我们使用了 6 层。
    \item \textbf{Persistence ($b$)}:振幅衰减系数,通常取 0.5。即频率越高,对整体高度的贡献越小(表现为岩石表面的小凹凸)。
    \item \textbf{Lacunarity}:频率增长系数,通常取 2.0。即每一层细节的密度翻倍。
\end{itemize}

通过这种方式,第一层噪声决定了山脉的大致走向,第二层决定了山峦的起伏,第三层决定了岩石的碎裂... 最终合成出极具真实感的地形。

\newpage

\section{系统架构设计}

\subsection{设计原则}
本项目采用 \textbf{面向对象编程 (OOP)} 思想,遵循 SOLID 设计原则,特别是单一职责原则。我们将渲染引擎的功能模块化,使其解耦,便于维护和扩展。

系统总体采用“分层架构”,从底向高层提供服务。

\subsection{模块划分}
系统主要由以下三大模块组成:
\begin{enumerate}
    \item \textbf{核心层 (Core Layer)}:负责底层的窗口管理、输入处理和时间管理。
    \item \textbf{渲染层 (Graphics Layer)}:封装 OpenGL API,提供 Shader、Texture、Mesh、Camera 等抽象类。
    \item \textbf{世界层 (World Layer)}:负责游戏逻辑,包括地形生成、对象更新、物理模拟等。
\end{enumerate}

\subsection{类图与关系}
以下是系统的核心类及其关系描述:

\begin{itemize}
    \item \textbf{Application}:单例模式,程序入口,管理主循环。
    \item \textbf{Window}:拥有 GLFWwindow 指针,管理上下文。
    \item \textbf{Input}:静态类,处理 `glfwSetKeyCallback` 传入的事件。
    \item \textbf{Renderer}:持有 Shader 和 Camera,负责具体的绘制命令。
    \item \textbf{World}:包含 `InfiniteTerrain` 和 `Plane` 对象。
\end{itemize}

\subsection{核心类详细设计}

\subsubsection{Window 类}
该类封装了 GLFW 库的所有操作,向应用层屏蔽了操作系统窗口管理的复杂性。
\begin{itemize}
    \item \textbf{初始化}:在构造函数中调用 `glfwInit()`,设置 OpenGL 版本号 3.3 Core Profile,设置 MSAA(多重采样抗锯齿)等 Hint。
    \item \textbf{窗口创建}:调用 `glfwCreateWindow()` 创建物理窗口。
    \item \textbf{上下文管理}:调用 `glfwMakeContextCurrent()` 绑定当前线程。
    \item \textbf{事件循环}:提供 `shouldClose()` 和 `swapBuffers()` 接口供主循环调用。
    \item \textbf{回调机制}:注册 `framebuffer\_size\_callback` 以处理窗口大小变化,注册鼠标键盘回调函数。
\end{itemize}

\subsubsection{Shader 类}
这是本项目中使用频率最高的工具类。现代 OpenGL 编程是基于可编程管线的,因此一个不仅能加载代码,还能方便地设置 Uniform 变量的着色器类至关重要。

该类实现了以下功能:
\begin{enumerate}
    \item \textbf{读取文件}:从磁盘读取 Vertex 和 Fragment Shader 源码流。支持 C++ 的 `ifstream` 和 `stringstream` 操作。
    \item \textbf{编译与检错}:调用 `glCompileShader` 并检查 `GL\_COMPILE\_STATUS`。若编译失败,输出显卡驱动返回的具体的行号和错误信息 Log。
    \item \textbf{链接}:调用 `glLinkProgram` 生成着色器程序对象。
    \item \textbf{Uniform 封装}:提供 `setBool`, `setInt`, `setFloat` 等基础类型,以及 `setVec3`, `setMat4` 等 GLM 数学类型的重载函数,内部调用 `glUniform...` 系列 API。这大大简化了主逻辑代码,例如:`shader.setMat4("view", camera.GetViewMatrix())`。
\end{enumerate}

\subsubsection{Camera 类}
摄像机类实现了 3D 漫游功能。为了支持更自由的控制,我们维护了以下状态向量:
\begin{itemize}
    \item `Position`:摄像机位置。
    \item `Front`:摄像机前方向量。
    \item `Up`:摄像机上方向量。
    \item `Right`:摄像机右方向量。
    \item `WorldUp`:世界坐标系的上方(通常是 $(0,1,0)$)。
    \item `Yaw`, `Pitch`:欧拉角。
\end{itemize}
每帧调用 `updateCameraVectors()` 重新计算正交基向量,并提供 `GetViewMatrix()` 返回 LookAt 矩阵。这一数学抽象使得我们无需关心底层的矩阵运算,只需关注摄像机的位置和角度。

\newpage

\section{无限地形系统实现}

本章将深入探讨 Skyscape 的核心——无限地形系统的实现细节。

\subsection{区块化 (Chunking) 策略}
要在 16GB 的内存中渲染一个“无限”的世界是不可能的。核心思想是 \textbf{动态加载 (Dynamic Loading)},即只渲染玩家周围可见区域的世界。我们将世界平面划分为大小为 $ChunkSize \times ChunkSize$ 的网格区块(Chunk)。

在每一帧的 `Update` 阶段,系统执行以下逻辑:
\begin{enumerate}
    \item 计算摄像机当前所在的逻辑区块坐标 $(cx, cz)$。
    \item 以 $(cx, cz)$ 为中心,遍历周围 $R$ 半径(Render Distance)内的所有区块坐标。
    \item 检查这些区块是否存在于 `std::map<ChunkKey, Chunk>` 数据结构中。这里的 `ChunkKey` 是一个 `pair<int, int>` 类型的哈希键。
    \item 如果不存在,则立即\textbf{生成}该区块(计算顶点、构建网格、上传 GPU)。
    \item 检查已存在的区块,如果距离摄像机超过了卸载半径 $R_{unload}$,则\textbf{销毁}该区块(释放 `glDeleteBuffers`,从 Map 中移除)。
\end{enumerate}

这种机制确保了无论玩家飞行多远,内存中始终只保留常数数量(例如 $16 \times 16 = 256$ 个)的区块,从而实现了理论上的无限漫游。

\begin{algorithm}[H]
\caption{Chunk 管理逻辑伪代码}
\begin{algorithmic}[1]
\STATE Let $Dictionary$ be the map of active chunks
\STATE Let $Pos$ be camera position
\STATE $CX \leftarrow \lfloor Pos.x / SIZE \rfloor$
\STATE $CZ \leftarrow \lfloor Pos.z / SIZE \rfloor$
\FOR{$dx = -ViewDist$ to $ViewDist$}
    \FOR{$dz = -ViewDist$ to $ViewDist$}
        \STATE $Key \leftarrow (CX+dx, CZ+dz)$
        \IF{$Key$ not in $Dictionary$}
            \STATE $NewChunk \leftarrow \text{GenerateMesh}(Key)$
            \STATE $Dictionary.\text{add}(Key, NewChunk)$
        \ENDIF
    \ENDFOR
\ENDFOR
\FORALL{$Chunk$ in $Dictionary$}
    \IF{$\text{Distance}(Chunk, Pos) > UnloadDist$}
        \STATE $\text{OpenGL::Delete}(Chunk.VAO)$
        \STATE $Dictionary.\text{remove}(Chunk)$
    \ENDIF
\ENDFOR
\end{algorithmic}
\end{algorithm}

\subsection{地形数据的构建}
每个 Chunk 的生成过程涉及大量的计算,是 CPU 的主要负载来源。具体步骤如下:
\begin{enumerate}
    \item \textbf{申请内存}:为顶点数组(Vertices)和索引数组(Indices)分配堆内存空间。
    \item \textbf{遍历网格点}:对于本地坐标 $0$ 到 $SIZE$ 的每个点 $(x, z)$:
        \begin{enumerate}
            \item 计算世界坐标 $WorldX = ChunkX \times SIZE + x$。这保证了不同 Chunk 的网格在世界空间中是连续对齐的。
            \item 调用 `Noise(WorldX, WorldZ)` 函数获取高度 $Y$。
            \item 估算法线向量(Normal),用于光照计算。
            \item 根据 $Y$ 值计算颜色(Color),作为顶点的属性传入 Shader。
            \item 将 $(x, y, z, nx, ny, nz, r, g, b)$ 压入顶点数组。
        \end{enumerate}
    \item \textbf{构建索引}:生成绘制 GL\_TRIANGLES 所需的索引数据。对于每个格子(Grid),我们需要绘制两个三角形。
    \item \textbf{上传 GPU}:
        \begin{lstlisting}[language=C++]
        glGenVertexArrays(1, &VAO);
        glBindVertexArray(VAO);
        glGenBuffers(1, &VBO);
        glBufferData(..., vertices, GL_STATIC_DRAW);
        // 设置各属性指针 (Layout 0: Pos, Layout 1: Normal, Layout 2: Color)
        glVertexAttribPointer(...); 
        \end{lstlisting}
\end{enumerate}

\subsection{法线计算细节}
在光照计算中,顶点的法线(Normal)至关重要。对于解析曲面(如球体),我们可以通过求导得到精确法线。但对于噪声生成的随机地形,我们需要使用数值方法——\textbf{有限差分法}。

具体来说,对于点 $P(x, z)$,我们查询其相邻点的高度:
\begin{align}
    h_L &= \text{Noise}(x-1, z) \\
    h_R &= \text{Noise}(x+1, z) \\
    h_D &= \text{Noise}(x, z-1) \\
    h_U &= \text{Noise}(x, z+1) 
\end{align}
然后构建两个切向量:
\begin{align}
    \vec{V_1} &= (2, h_R - h_L, 0) \\
    \vec{V_2} &= (0, h_U - h_D, 2)
\end{align}
最后通过叉乘得到法线:
\begin{equation}
    \vec{N} = \text{normalize}(\vec{V_2} \times \vec{V_1})
\end{equation}

这里有一个关键的实现细节:\textbf{跨区块边界的处理}。
如果在计算 Chunk 边缘顶点的法线时,只使用 Chunk 内部的数据,那么在两个 Chunk 的交界处,由于缺乏外部邻居的信息,法线会计算错误,导致光照出现明显的折痕(Seams)。
解决方法是:在计算法线时,\textbf{不查询} `vertices` 数组,而是\textbf{重新调用}全局的 `Noise` 函数查询 $x-1$ 或 $x+1$ 的高度。因为 `Noise` 函数是确定性的(Deterministic),无论在哪个 Chunk 调用,对于同一个世界坐标 $GetWorldHeight(x, z)$ 返回值永远一样。这样就保证了跨区块的法线连续性,使得整个世界看起来像是一个整体。

\subsection{过程化纹理着色}
为了表现丰富的地貌,我们没有使用静态纹理(Texture Splatting),而是采用过程化着色。这既节省了纹理内存,又使得地形风格统一。
在 Fragment Shader 中,我们根据传入的 `WorldPos.y`(高度)进行判断:
\begin{itemize}
    \item $Y < 0$: 蓝色(Deep Water),模拟深海。
    \item $0 < Y < 2$: 黄色(Sand Beach),模拟沙滩。
    \item $2 < Y < 30$: 绿色(Grass),模拟平原和森林。
    \item $30 < Y < 50$: 灰褐色(Rock),模拟高山岩石。
    \item $Y > 50$: 白色(Snow),模拟山顶积雪。
\end{itemize}
为了避免颜色分界过于生硬(Hard Edge),我们使用了 `smoothstep` 或 `mix` 函数,在两个生物群落(Biome)的交界处(如高度 30 到 35 之间)进行线性插值混合。这创造出了自然的过渡带。

\newpage

\section{渲染与着色技术}

本章主要介绍运行在 GPU 上的着色器代码(GLSL)及其实现的高级视觉效果。

\subsection{着色器管线}

\subsubsection{顶点着色器 (Vertex Shader)}
顶点着色器的主要任务是坐标变换。它将输入的模型空间坐标 `aPos` 一步步变换到裁剪空间。
\begin{lstlisting}[language=GLSL]
#version 330 core
layout (location = 0) in vec3 aPos;
layout (location = 1) in vec3 aNormal;
layout (location = 2) in vec3 aColor;

uniform mat4 model;
uniform mat4 view;
uniform mat4 projection;

out vec3 FragPos;
out vec3 Normal;
out vec3 Color;

void main() {
    // 关键:计算世界坐标,用于光照和雾效计算
    FragPos = vec3(model * vec4(aPos, 1.0));
    
    // 关键:法线矩阵变换
    // 当存在非均匀缩放时,直接使用 model 矩阵会破坏法线的垂直性
    // 必须使用 model 的逆转置矩阵 (Normal Matrix)
    Normal = mat3(transpose(inverse(model))) * aNormal;
    
    Color = aColor; // 将顶点颜色透传给片元着色器进行插值
    gl_Position = projection * view * vec4(FragPos, 1.0);
}
\end{lstlisting}

\subsubsection{片段着色器 (Fragment Shader)}
片段着色器负责最终的像素颜色计算。这里实现了 Phong 光照模型和雾效混合。
首先,我们归一化法线和光线方向,计算 Ambient, Diffuse, Specular 三个分量。
\begin{lstlisting}[language=GLSL]
// 漫反射计算
vec3 norm = normalize(Normal);
vec3 lightDir = normalize(lightPos - FragPos);
float diff = max(dot(norm, lightDir), 0.0);
vec3 diffuse = diff * lightColor;
\end{lstlisting}
然后,将计算出的光照结果与顶点的基础颜色 `Color` 相乘,得到物体本来的颜色。

\subsection{距离雾 (Distance Fog)}
在无限世界中,如果直接让远处物体突然消失(Clipping),视觉体验会非常糟糕。为了掩盖 Chunk 动态加载时的“突然出现”问题,并增强场景的深度感,我们实现了线性雾(Linear Fog)。

基本原理是:物体离摄像机越远,其颜色就越接近背景色(天空色)。
在 Fragment Shader 中:
\begin{lstlisting}[language=GLSL]
// 计算片元到摄像机的欧几里得距离
float dist = length(ViewPos - FragPos);

// 定义雾的起止范围
float fogStart = 500.0; // 500单位内清晰
float fogEnd = 1500.0;  // 1500单位处完全不可见

// 计算混合因子 [0, 1]
float fogFactor = clamp((fogEnd - dist) / (fogEnd - fogStart), 0.0, 1.0);

// 混合物体颜色与雾颜色
vec3 finalColor = mix(SkyColor, ObjectColor, fogFactor);
\end{lstlisting}
当距离超过 1500 时,`fogFactor` 变为 0(注意上述公式分母顺序,或者使用 mix 的反向参数),物体完全变为天空色,从而实现了完美的视觉融合。玩家只会感觉远处雾蒙蒙的,而不会注意到地形是在生成中。

\subsection{天空盒 (Skybox)}
天空盒通过 `samplerCube` 采样器实现。我们加载了 6 张天空纹理(Right, Left, Top, Bottom, Back, Front)形成一个立方体贴图。
渲染天空盒的一个重要技巧是:\textbf{移除视角的位移}。
即在 C++ 端传递 View 矩阵时,我们将 $4 \times 4$ 矩阵转换为 $3 \times 3$ 矩阵再转回,从而丢弃平移分量:
\begin{lstlisting}[language=C++]
glm::mat4 view = glm::mat4(glm::mat3(camera.GetViewMatrix()));
\end{lstlisting}
这就去掉了平移分量,使得无论玩家怎么飞行,天空盒永远看起来在无穷远处,不会被追上。

此外,为了性能优化,我们最后渲染天空盒,并利用 \textbf{Early-Z} 测试。我们强制天空盒的顶点 $Z$ 值为 $1.0$ (NDC 最大深度):
\begin{lstlisting}[language=GLSL]
// Vertex Shader trick
vec4 pos = projection * view * vec4(aPos, 1.0);
// xyww 变换后,z 分量会被 w 分量替换
// 在透视除法后,z/w = w/w = 1.0,即最远深度
gl_Position = pos.xyww; 
\end{lstlisting}
然后设置深度测试函数为 `glDepthFunc(GL\_LEQUAL)`。这样只有当屏幕上没有其他物体遮挡(即该像素未被地形覆盖)时,才会绘制天空像素。这比先画天空盒再画地形节省了大量的 Fragment Shader 开销。

\newpage

\section{物理与交互系统}

\subsection{飞行控制逻辑}
本系统的飞行控制并非简单的移动摄像机(如 NoClip 模式),而是基于一个简化的空气动力学模型,赋予玩家驾驶真实飞机的感觉。

\subsubsection{坐标系定义}
飞机的状态由位置 $P$、旋转欧拉角 $(Yaw, Pitch, Roll)$ 和速度标量 $Speed$ 决定。
每一帧,我们首先根据当前的 $Yaw$ 和 $Pitch$ 计算飞机的朝向向量 $D_{front}$:
\begin{align}
    D_x &= \cos(Pitch) \cdot \cos(Yaw) \\
    D_y &= \sin(Pitch) \\
    D_z &= \cos(Pitch) \cdot \sin(Yaw)
\end{align}
注意这里与摄像机不同,飞机需要显式地维护一个模型矩阵,以便渲染飞机自身的模型。

\subsubsection{输入响应}
为了模拟真实操作,我们将用户的按键输入映射为对“加速度”和“角速度”的改变,而不是直接改变位置。
\begin{table}[H]
\centering
\caption{控制键位映射表}
\begin{tabular}{|c|c|c|l|}
\hline
\textbf{输入设备} & \textbf{变量} & \textbf{物理量} & \textbf{功能描述} \\ \hline
键盘 W / S & Speed & 推力 (Thrust) & 按住加速,松开缓慢减速(空气阻力) \\ \hline
键盘 A / D & Yaw & 偏航力矩 & 控制垂直尾翼,改变水平朝向 \\ \hline
键盘 Shift & Speed & 加力 (Boost) & 开启加力燃烧室,极速提升 2 倍 \\ \hline
鼠标 X 轴 & Roll & 滚转力矩 & 控制副翼,改变机身滚转角 \\ \hline
鼠标 Y 轴 & Pitch & 俯仰力矩 & 控制升降舵,改变机头俯仰角 \\ \hline
\end{tabular}
\end{table}

代码逻辑示例:
\begin{lstlisting}[language=C++]
if (keys[GLFW_KEY_W]) speed += acceleration * deltaTime;
if (keys[GLFW_KEY_S]) speed -= acceleration * deltaTime;
// 阻力模拟
speed *= 0.99f; 
// 位置更新
Position += FrontVector * speed * deltaTime;
\end{lstlisting}

\subsection{摄像机跟随算法}
为了提供类似 3A 游戏的驾驶手感,摄像机实现了一个带有 \textbf{惯性滞后 (Inertia Lag)} 的跟随系统。
不再是将摄像机硬绑定在飞机坐标系上(如简单的父子层级关系),而是每一帧计算一个“理想目标位置”:
\begin{equation}
    Target = PlanePos - PlaneFront \cdot Dist + PlaneUp \cdot Height
\end{equation}
这个 $Target$ 位置始终位于飞机尾部后上方固定距离。
然后使用插值算法让摄像机当前位置平滑地逼近目标位置:
\begin{lstlisting}[language=C++]
// 这里的 SmoothFactor 决定了相机的“软”度
// 值越小,相机越滞后,速度感越强;值越大,相机越死板
CameraPos = glm::mix(CameraPos, Target, SmoothFactor * deltaTime);

// 同时也让摄像机的朝向平滑地转向飞机前方
CameraFront = glm::mix(CameraFront, PlaneFront, RotationSmooth * deltaTime);
\end{lstlisting}
这种处理使得当飞机急加速或急转弯时,摄像机会产生自然的延迟,极大地增强了速度感和动态感。

\newpage

\section{测试、分析与优化}

\subsection{开发环境}
\begin{itemize}
    \item \textbf{硬件}:AMD Ryzen 7 5800H (8核16线程), NVIDIA RTX 3060 Laptop (6GB VRAM), 16GB DDR4 3200MHz RAM
    \item \textbf{软件}:Windows 11 Professional 22H2
    \item \textbf{工具链}:Visual Studio Code, CMake 3.20+, GCC (MinGW-w64) 12.2.0
\end{itemize}

\subsection{性能分析}
为了评估系统的性能,我们在 1920x1080 全屏分辨率下进行了多项压力测试。

\subsubsection{帧率测试}
\begin{table}[H]
\centering
\caption{不同视距下的性能表现}
\begin{tabular}{|c|c|c|c|c|}
\hline
\textbf{视距 (Render Dist)} & \textbf{区块数量} & \textbf{顶点总数 (估算)} & \textbf{平均 FPS} & \textbf{显存占用} \\ \hline
4 Chunks & 81 & 0.3M & 300+ & 120MB \\ \hline
8 Chunks & 289 & 1.2M & 140 & 350MB \\ \hline
12 Chunks & 625 & 2.6M & 85 & 700MB \\ \hline
16 Chunks & 1089 & 4.5M & 45 & 1.2GB \\ \hline
\end{tabular}
\end{table}

分析可见,在默认视角(视距 8 Chunks)下,帧率稳定在 140 FPS 以上,完全满足流畅运行的需求。当视距增加到 16 Chunks 时,Chunk 数量变为原来的 4 倍,帧率下降至 45 FPS。瓶颈主要在于 `Draw` 调用次数(Draw Calls)过多,CPU 提交命令的开销成为了限制因素。未来可以通过 \textbf{GPU Instance} 技术或者合批(Batching)来优化。

\subsubsection{生成时延}
地形生成算法完全在 CPU 端运行。经过测算,生成一个 $64 \times 64$ 的 Chunk(包含 4096 个顶点和法线计算)大约耗时 0.5ms。由于每帧摄像机移动距离有限,通常只需要生成边缘的 3-5 个新 Chunk,总耗时 < 3ms,不会造成画面卡顿。

\subsection{遇到的问题与解决方案}

在开发过程中,我们遇到了两个经典的图形学难题。

\subsubsection{Z-Fighting (深度冲突)}
\begin{warningbox}[现象描述]
当飞机飞至高空(Y > 500)俯瞰地面时,远处的山峦纹理出现疯狂闪烁,就像两个物体在不断争抢显示权。
\end{warningbox}
\textbf{原因分析}:透视投影矩阵的非线性性质导致深度缓冲区(Depth Buffer)的精度分布极不均匀。大部分精度都集中在近平面附近,而远平面处的精度极低。当 $zNear = 0.1$ 时,在距离 1000 的地方,浮点数的精度误差已经超过了两个相邻像素的深度差。
\textbf{解决方案}:
1. 增大近平面距离:将 `zNear` 从 0.1 调整为 1.0。这极大地提高了远处的深度精度。
2. 配合距离雾:让极远处的物体被雾遮挡,即使发生 Z-Fighting 也看不见。

\subsubsection{Floating Point Jitter (大坐标抖动)}
\begin{warningbox}[现象描述]
向一个方向飞行数十分钟,坐标达到 $100,000$ 量级后,飞机模型开始剧烈抖动,地形出现裂缝,物理碰撞失效。
\end{warningbox}
\textbf{原因分析}:`float` 类型(32位 IEEE 754)只有 23 位尾数,有效十进制位数约 7 位。当坐标值为 $100,000$ 时,最小可表示的精度间隔(Epsilon)变为 $0.01$ 甚至更高。这意味着顶点不能平滑移动,只能在格点间跳跃。
\textbf{解决方案(思路)}:由于时间限制,本项目暂未完全实现,但提出了 \textbf{浮动原点 (Floating Origin)} 的设计方案。即每当摄像机距离原点超过阈值(如 10000 单位),就将整个世界的所有物体(包括 Chunk 坐标、飞机位置)反向平移,重置原点到摄像机脚下 $(0,0,0)$。这样始终保持浮点数在精度最高的区间运算。

\newpage

\section{总结与展望}
\subsection{工作总结}
本系统经过一个月的开发与迭代,成功实现了一个基于 OpenGL 的轻量级飞行模拟引擎。主要成果包括:
\begin{enumerate}
    \item \textbf{渲染架构}:搭建了完整的 C++/OpenGL 渲染框架,封装了 Window, Shader, Camera, Texture 等核心组件。
    \item \textbf{地形生成}:实现了高效的无限地形生成系统,解决了动态加载与内存管理问题。
    \item \textbf{视觉效果}:实现了基于 Shader 的多重生物群落纹理混合、冯氏光照、距离雾和天空盒,画面风格统一且具有美感。
    \item \textbf{交互体验}:实现了手感良好的物理飞行控制和惯性摄像机,提供了沉浸式的漫游体验。
\end{enumerate}

\subsection{未来展望}
虽然项目已达成了基本目标,但距离商业级引擎仍有差距,未来可在以下方向改进:
\begin{itemize}
    \item \textbf{LOD (Level of Detail)}:引入四叉树(Quadtree)算法,根据距离动态改变 Chunk 的网格密度(近处高模,远处低模),以支持超远视距(如 32 Chunks 以上)而不降低帧率。
    \item \textbf{阴影渲染}:目前主要靠光照明暗来表现立体感,缺乏真实的投影。未来可实现 Cascade Shadow Maps (CSM) 以支持大范围地形的动态阴影。
    \item \textbf{更高级的噪声}:使用 Simplex Noise 替代 Perlin Noise 以提高计算效率,并引入水力侵蚀(Hydraulic Erosion)算法生成更真实的河流和沟壑。
    \item \textbf{计算着色器}:将地形生成逻辑放入 Compute Shader,利用 GPU 的并行能力实现毫秒级生成,彻底释放 CPU。
\end{itemize}

这次大作业不仅让我掌握了图形学的核心算法,更让我深刻体会到了数学之美。当你看着那一行行抽象的公式——分形噪声、点积、叉乘、矩阵变换——在屏幕上化作连绵起伏的群山、波光粼粼的湖面和自由翱翔的飞机时,那种创造世界的成就感是无与伦比的。这正是计算机图形学的魅力所在:用逻辑构建世界,用代码诠释自然。

\end{document}
